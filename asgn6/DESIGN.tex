\documentclass[11pt]{article}
\title{cse13s asgn6 DESIGN.pdf}
\author{Lucas Lee; CruzID: luclee}
\date{2/18/2022}
\begin{document}\maketitle
\section{Program Details}\label{ss:details}
Program implements Huffman Encoder and Decoder, as well as structures for stacks and queues.
\section{Huffman Encoder}\label{ss:encoder}
encode():
\begin{enumerate}
\item -h prints out a help message
\item -i (infile) sets the file that is to be encoded. Default is stdin
\item -o (outfile) sets the file to be encoded to. Default is stdout
\item -v prints uncompressed file size, compressed file size, and space saving to stderr
\end{enumerate}
To Huffman encode the file:
\begin{enumerate}
\item count the number of each unique symbol in the file (most likely use a struct)
\item use priority queue to make a Huffman tree
\item use stack of bits to traverse Huffman tree. Make a code table to index each symbol 
\item encode the Huffman tree to the file using post-order traversal
\item for each symbol in the input file, write the corresponding code to output
\end{enumerate}
decode():
\begin{enumerate}
\item -h prints help message
\item -i (infile) specifies the file to decode. Default is stdin
\item -o (outfile) specifies the file to decode to. Default is stdout
\item -v prints compressed file size, decompressed file size, space saving.
\end{enumerate}
Algorithm steps for decoding:
\begin{enumerate}
\item read Huffman tree from the input file.
\item use a stack of nodes to reconstruct Huffman tree
\item read each bit of the input file, reading a 0 when going down a left link (lower value) and reading a 1 when going down the right link (higher value)
\item when reaching an end node, emit the symbol and start traversing the tree again from the beginning and not going the same path
\end{enumerate}
\section{Nodes}\label{ss:Nodes}
We will make an ADT for nodes which will act as linked lists needed for Huffman trees.
\begin{enumerate}
\item make Node struct that has pointers to left and right Nodes, as well as variables to record the symbol and frequency of the variable
\end{enumerate}
We will use the following functions in creating the Nodes:
\begin{enumerate}
\item NodeCreate(uint8  symbol, uint64 freq)
	\begin{enumerate}
	\item constructs node. Allocates space and initializes the variables to the values in the parameters
	\end{enumerate}
\item NodeDelete(Node N)
	\begin{enumerate}
	\item deletes Node struct N. Free memory allocated when creating node. 
	\end{enumerate}
\item NodeJoin(Node left, Node right)
	\begin{enumerate}
	\item creates a new node that returns a Node with the left and right value combined.
	\item frequency will be the sum of left and right frequency
	\item symbol will be \textdollar
	\end{enumerate}
\item NodePrint(Node n)
	\begin{enumerate}
	\item print node n for debugging purposes.
	\end{enumerate}
\end{enumerate}
\section{Priority Queue}\label{ss:Prio}
We will also need to implement a priority queue struct for the enqueue.
\begin{enumerate}
\item pqCreate(uint32 capacity)
	\begin{enumerate}
	\item constructs a priority queue making the max capacity specified in the parameter
	\item malloc using capacity
	\end{enumerate}
\item pqDelete(PriorityQueue q)
	\begin{enumerate}
	\item destroys Priority Queue q.
	\item set pointer to NULL after freeing the memory from q
	\end{enumerate}
\item bool pqEmpty(PriorityQueue q)
	\begin{enumerate}
	\item return true for empty q, false otherwise
	\end{enumerate}
\item bool pqFull(PriorityQueue q)
	\begin{enumerate}
	\item return true if full q, false otherwise
	\end{enumerate}
\item uint32 pqSize(PriorityQueue q)
	\begin{enumerate}
	\item variable to keep track of size of q
	\item for each index in q, index size variable
	\item return the size
	\end{enumerate}
\item bool enqueue(PriorityQueue q, Node n)
	\begin{enumerate}
	\item put node into priority queue, return false if priority queue is full beforehand, and true if successfully enqueued
	\end{enumerate}
\item bool dequeue(PriorityQueue q, Node n)
	\begin{enumerate}
	\item dequeue node n from q, pass it back with a double pointer.
	\item return false if priority queue is empty and true if successfully dequeued
	\end{enumerate}
\item pqPrint(PriorityQueue q)
	\begin{enumerate}
	\item for each element in q, print it out for testing purposes
	\item this function acts as a debug function
	\end{enumerate}
\end{enumerate}
\section{Huffman Codes}\label{ss:codes}
We will create a Code struct:
\begin{enumerate}
\item Code codeInit()
	\begin{enumerate}
	\item initialize code by setting variable top to 0 and zeroing out bits
	\end{enumerate}
\item uint32 codeSize(Code c)
	\begin{enumerate}
	\item return size of code. This is the number of bits pushed to Code c
	\end{enumerate}
\item bool codeEmpty(Code c)
	\begin{enumerate}
	\item return true if Code c is empty, false otherwise
	\end{enumerate}
\item bool codeFull(Code c)
	\begin{enumerate}
	\item return true if Code c is full, false otherwise
	\end{enumerate}
\item bool codeSetBit(Code c, uint32 i)
	\begin{enumerate}
	\item set Code c at index i to 1. If i is out of range return false. Return true if successful
	\end{enumerate}
\item bool codeClrBit(Code c, uint32 i)
	\begin{enumerate}
	\item clear Code c at index i by setting it to 0. If i is out of range return false. Return true if successful otherwise
	\end{enumerate}
\item bool codeGetBit(Code c, uint32 i)
	\begin{enumerate}
	\item get bit in Code c at index i. If i is out of range or index i is equal to 0 return false. Return true only if index i is equal to 1
	\end{enumerate}
\item bool codePushBit(Code c, uint8 bit)
	\begin{enumerate}
	\item push bit to Code c
	\item return false if Code is full before pushing bit
	\item return true if bit is successfully pushed
	\end{enumerate}
\item bool codePopBit(Code c, uint8 bit)
	\begin{enumerate}
	\item pop bit from Code c by passing the value back into pointer bit.
	\item return false if Code c is empty, and true if bit is successfully popped
	\end{enumerate}
\item codePrint(Code c)
	\begin{enumerate}
	\item debug function that prints out elements in c
	\item used to determine if push and pop are successful or not
	\end{enumerate}
\end{enumerate}

\section{I/O}\label{ss:IO}
We will need to make an IO file for reading and writing bytes
\begin{enumerate}
\item int readBytes(int infile, uint8 buf, int nbytes)
	\begin{enumerate}
	\item infile parameter is the number of bytes read from the file descriptor
	\item loops calls to read() until nbytes are read from the file descriptor or there are no more bytes in the file to read (eof)
	\end{enumerate}
\item int writeBytes(int outfile, uint8 buf, int nbytes)
	\begin{enumerate}
	\item loop calls to write()
	\item loop until nbytes are written, all bytes of buf are written, or no bytes are written
	\end{enumerate}
\item bool readBit(int infile, uint8 bit)
	\begin{enumerate}
	\item read bytes into a buffer and read each bit from it one at a time
	\item create a buffer of bytes to track which bit to return to the bit pointer.
	\item buffer stores BLOCK number of bytes and returns false if there are no bits that can be read. Returns true if there are still bits to read.
	\end{enumerate}
\item writeCode(int outfile, Code c)
	\begin{enumerate}
	\item use bit buffer similar to readBit.
	\item write each bit in Code c to buffer starting from 0th bit
	\item when buffer has BLOCK bytes, write buffer contents to outfile
	\end{enumerate}
\item flushCodes(int outfile)
	\begin{enumerate}
	\item this function makes sure that there are no bits leftover after writeCode is called, so this writes out any left out bits in the buffer
	\end{enumerate}
\end{enumerate}
\section{Stacks}\label{ss:stacks}
We need to make a stack of nodes to reconstruct the Huffman tree
\begin{enumerate}
\item stackCreate(uint32 capacity)
	\begin{enumerate}
	\item constructor for stack that sets max nodes to capacity
	\end{enumerate}
\item stackDelete(Stack s)
	\begin{enumerate}
	\item destroys stack, sets pointer to NULL after freeing allocated memory from stack
	\end{enumerate}
\item bool stackEmpty(Stack s)
	\begin{enumerate}
	\item return true if s is empty, false if not
	\end{enumerate}
\item bool stackFull(Stack s)
	\begin{enumerate}
	\item return true if stack is full, false if not
	\end{enumerate}
\item uint32 stackSize(Stack s)
	\begin{enumerate}
	\item return number of nodes in s
	\end{enumerate}
\item bool stackPush(Stack s, Node n)
	\begin{enumerate}
	\item push n onto s. Return false if stack is full beforehand and true if push is successful
	\end{enumerate}
\item bool stackPop(Stack s, Node n)
	\begin{enumerate}
	\item pop n from s, and pass popped n back into double pointer n. Return false if stack was empty before pop and true if pop was successful
	\end{enumerate}
\item stackPrint(Stack s)
	\begin{enumerate}
	\item prints out the contents of s
	\end{enumerate}
\end{enumerate}
\section{Huffman code module}\label{ss:module}
\begin{enumerate}
\item Node buildTree(uint64 hist(static ALPHABET))
	\begin{enumerate}
	\item builds node tree based on histogram given ALPHABET indices corresponding to different symbols
	\item returns root node for the tree
	\end{enumerate}
\item buildCodes(Node root, Code table(static ALPHABET))
	\begin{enumerate}
	\item makes a code table for each symbol in the Huffman tree, copies the codes to table, which has ALPHABET indices for each symbol
	\end{enumerate}
\item dumpTree(int outfile, Node root)
	\begin{enumerate}
	\item writes to outfile L followed by byte of symbol for a leaf node and an I for interior nodes
	\item do not write the symbol for interior nodes
	\end{enumerate}
\item Node rebuildTree(uint16 nbytes, uint8 treeDump(static nbytes))
	\begin{enumerate}
	\item reconstructs Huffman tree using treeDump
	\item length of treeDump is nbytes. Returns root node of reconstructed tree
	\end{enumerate}
\item deleteTree(Node root)
	\begin{enumerate}
	\item deconstructs Huffman tree
	\item post-order traversal to free each node in the tree
	\item set pointer root to NULL after freeing each node
	\end{enumerate}
\end{enumerate}
\section{File Encoder}\label{ss:encoder}
Steps to implement the Huffman encoder
\begin{enumerate}
\item read infile to make a histogram using struct
\item increment index 0 and 255 of histogram array to make sure that there are at least 2 elements
\item buildTree() to make Huffman tree using histogram
	\begin{enumerate}
	\item make a priority queue and add each symbol that has a non 0 frequency to the queue and make a node of it
	\item while there are two or more nodes in priority queue, dequeue two nodes, one left and one right
	\item join these two dequeued nodes and queue the now created parent node
	\item frequency of parent node = sum of children frequencies
	\item when there is one node left in the queue, that is the root of the tree
	\end{enumerate}
\item make a code table by stepping through the Huffman tree created
	\begin{enumerate}
	\item create Code c using codeInit() starting at root of the tree
	\item if current node is a leaf, c is the path to a node and has the node's symbol. Save the code to the table
	\item if it is not a leaf, push 0 to Code c and go down the left link
	\item once returning from the left, pop bits from c, push 1 to c and go down the right links
	\item when returning from right, pop bit from c.
	\end{enumerate}
\item make a header supplied in header.h
\item write header to outfile
\item write Huffman tree to outfile using dumpTree
\item write corresponding symbol code to outfile by reading infile
\item flush remaining buffered bits using flushCodes
\item close both files
\end{enumerate}
\section{Decoder specifics}\label{ss:decoder}
\begin{enumerate}
\item read header from infile and its magic number
\item if magic number does not match MAGIC in defines.h, then quit program
\item set outfile permissions using fchmod() with given permissions in the header
\item read dumped tree from infile and reconstruct Huffman tree (rebuildTree())
	\begin{enumerate}
	\item treeDump = array containing dumped tree
	\item treeDump length = nbytes
	\item loop over contents from 0 to nbytes in treeDump
	\item if element of array is L (leaf), then the next element of the array is its symbol. Create a node using that symbol and push it to the stack
	\item if element of array is I (interior node), then pop the stack to get the right child of that node, and pop again to get the left child of the node.
	\item join the two popped nodes and join them to create a parent node and push it to the stack
	\item once there is one node left in treeDump, that is the root of the Huffman tree
	\end{enumerate}
\item read infile for each bit using readBit()
	\begin{enumerate}
	\item begin at root of the tree, if a 0 is read then go left of current node. If a 1 is read then go right of the current node
	\item If a leaf node is encountered (next node is NULL) then write that symbol to outfile and reset the current node back to the root of the tree
	\item repeat for each read bit and until the decoded symbols equal those of the original file size from the header
	\end{enumerate}
\item close infile and outfile
\end{enumerate}

\end{document}
