\documentclass[11pt]{article}
\title{cse13s asgn5 DESIGN.pdf}
\author{Lucas Lee; CruzID: luclee}
\date{2/4/2022}
\begin{document}\maketitle
\section{Program Details}\label{ss:details}
We will create three functions: one that creates a public and private key pair using large prime numbers, one that encrypts a file using a public key, and one that decrypts the file using the private key. We will use a randomizer to generate the large primes for the keys.

\section{functions and pseudocode}\label{ss:functions}
for each function that needs random states, declare global variable state.
\begin{enumerate}
\item randstateinit(seed)
	\begin{enumerate}
	\item set Mersenne Twister initializer to state
	\item set srandom to seed
	\item set randseed with variable state to random seed
	\end{enumerate}
\item randstateclear()
	\begin{enumerate}
	\item clear memory of state
	\end{enumerate}
\item powmod(out, base, exponent, mod)
	\begin{enumerate}
	\item create mpz temp variables to store parameters so they do not change
	\item v = 1
	\item p = base
	\item tempexpo
	\item while tempexpo > 0
		\begin{enumerate}
		\item if tempexpo is odd
			\begin{enumerate}
			\item v= (v x p) mod(mod)
			\end{enumerate}
		\item p = (p x p) mod(mod)
		\item tempexpo = tempexpo floor division by 2
		\end{enumerate}
	\item out = v
	\item clear mpz variables
	\end{enumerate}
\item isprime(n, iters)
	\begin{enumerate}
	\item return true if n is less than 4
	\item return false if n is even
	\item r = (n-1) / \(2^s r\). r should be odd
		\begin{enumerate}
		\item create mpz values s, r, temp to hold values in the above equation
		\item s = 0, r = 1, temp = n-1
		\item while temp is not 1
			\begin{enumerate}
			\item floor divide temp by 2
			\item s += 1
			\item if temp is odd
				\begin{enumerate}
				\item r = temp
				\item break loop
				\end{enumerate}
			\end{enumerate}
		\end{enumerate}
	\item for loop from 1 - iters
		\begin{enumerate}
		\item create mpz variables to store temporoary values for calculations
		\item a = random number from (2, n-2)
		\item y = powmod(a, r, n)
		\item if y is not 1 or n-1
			\begin{enumerate}
			\item j = 1
			\item while j less than or equals s-1 and  y not equals n -1
				\begin{enumerate}
				\item if y == 1, clear mpz vars and return
				\item j = j + 1
				\end{enumerate}
			\item if y not equals n - 1. clearn mpz vars and return false
			\end{enumerate}
		\end{enumerate}
	\item clear all mpz vars used in function
	\item return true
	\end{enumerate}
\item makeprime(p, bits, iters)
	\begin{enumerate}
	\item create mpz variables to store temporary values. bitcount, prime, randombits
	\item bitcount = bits
	\item bitcount = \(2^(bits)\)
	\item set p = bitcount
	\item set randombits = bits -1
	\item set prime to a random number from \(2^b + 2^b -1\) if b = bits
	\item test using isprime(p, iters)
		\begin{enumerate}
		\item keep randomizing prime to a number in the same range to test for each iteration iters
		\end{enumerate}
	\item clear mpz variables and p should be set to the prime number
	\end{enumerate}
\item gcd(d, a, b)
	\begin{enumerate}
	\item make temp variables so d, a, and b are not altered by function
	\item while tempb is not 0
		\begin{enumerate}
		\item t = tempb
		\item tempb = tempa mod tempb
		\item tempa = t
		\end{enumerate}
	\item d = tempa
	\item clear mpz vars
	\end{enumerate}
\item modinverse(i, a, n)
	\begin{enumerate}
	\item make temp mpz variables to store 
	\item r = n
	\item rp = a
	\item t = 0
	\item tp = 1
	\item while rp not equal to 0
		\begin{enumerate}
		\item q = \(r/rp\)
		\item r = rp
		\item rp = r - q x rp
		\item t = tp
		\item tp = t - q x tp
		\end{enumerate}
	\item if r greater than 1: i = 0
	\item if t less than 0: t = t + n
	\item i = t
	\item return i
	\end{enumerate}
\item rsamakepub(p, q, n, e, nbits, iters)
	\begin{enumerate}
	\item p, q = prime numbers, n = product of p and q, e = exponent
	\item makeprime() to make p and q
	\item log2(n) should be greater than nbits
	\item p bits in range (nbits/4, (3 x nbits) /4)
	\item q gets remaining bits from the calculation
	\item random number using random() and iters to check prime
	\item lambda(n) = lcm(p-1, q-1)
		\begin{enumerate}
		\item do this by calculating gcd(p-1, q-1) and comparing it to the product of p-1 and q-1
		\item lcm(p-1, q-1) = absolute value(p-1 * q-1) / gcd(p-1, q-1)
		\end{enumerate}
	\item get random numbers around nbits
	\item get the gcd of each random number to find lambda(n)
	\item while gcd(e, lcm) is not 1 (not coprime)
		\begin{enumerate}
		\item randomize e from 0 - nbits
		\end{enumerate}
	\item coprime number lambda(n) = public exponent e
	\item clear mpz variables
	\end{enumerate}
\item rsawritepub(n,e,s, char username, file *pbfile)
	\begin{enumerate}
	\item open pbfile for writing (if not already open)
	\item print n in hex with a newline
	\item print e in hex with a newline
	\item print s in hex with a newline
	\item print username with newline
	\item close pbfile
	\end{enumerate}
\item rsareadpub(n,e,s, char username, file *pbfile)
	\begin{enumerate}
	\item open pbfile for reading(if not already open)
	\item scan each line to read then into variables
	\item scan first line = n
	\item scan second line = e
	\item scan thrid line = s
	\item scan fourth line = username
	\item close pbfile
	\end{enumerate}
\item rsamakepriv(d,e,p,q)
	\begin{enumerate}
	\item d = private key to be created, e = public exponent, p and q = primes
	\item d = modinverse(e, lcm)
	\end{enumerate}
\item rsawritepriv(n,d,file *pvfile)
	\begin{enumerate}
	\item open pvfile for writing (if not already open)
	\item write n as a hexstring followed by newline
	\item write d as a hexstring followed by newline
	\item close pvfile
	\end{enumerate}
\item rsareadpriv(n,d, file *pvfile)
	\begin{enumerate}
	\item open pvfile for reading(if not already open)
	\item scan lines to assign values to variables
	\item n = scan first line
	\item d = scan second line
	\end{enumerate}
\item rsa encrypt(c,m,e,n)
	\begin{enumerate}
	\item use powermod to compute c
	\item \(c = m^e (mod n)\)
	\end{enumerate}
\item rsa encrypt file(file *infile, file *outfile, n, e)
	\begin{enumerate}
	\item encrypt in blocks from infile to outfile
	\item create block size \(k = \lfloor (log_2 (n) -1 / 8) \lfloor\)
	\item malloc to allocate array that can hold k bytes as a uint8
	\item set array at 0 to 0xFF
	\item while unprocessed bytes in infile (using fread - bytes may not be numbers, so scan wont work)
		\begin{enumerate}
		\item read k - 1 bytes from infile and place them into the allocated block array starting from 1 (fread)
		\item convert the read bytes including array(0) into mpzt m (use mpz import for this)
		\item encrypt m using rsa encrypt() and write it to outfile as a hexstring with a newline. 
		\end{enumerate}
	\item close files (unless closed in functions) and free array
	\end{enumerate}
\item rsa decrypt(m, c, d, n)
	\begin{enumerate}
	\item compute message m using powermod
	\item \(m = c^d (mod m)\)
	\end{enumerate}
\item rsa decrypt file(file infile, file outfile, n, d)
	\begin{enumerate}
	\item allocate memory for block size similar to encrypt file
	\item while unprocessed bytes in infile(use feof() to indicate when the end of the file is reached)
		\begin{enumerate}
		\item scan hexstring, save hexstring in variable.
		\item convert each hexstring back into bytes using mpzexport() j = number of bytes read 
		\item write out j-1 bytes starting from array(1) to outfile
		\end{enumerate}
	\item print newline to outfile for syntax
	\item free array and close files (unless closed in functions)
	\end{enumerate}
\item rsa sign(s,m,d,n)
	\begin{enumerate}
	\item calculate s by using power mod
	\item \(s = m^d (mod n)\)
	\end{enumerate}
\item rsa verify(m,s,e,n)
	\begin{enumerate}
	\item calculate t by using power mod
	\item var t = \(s^e (mod n)\)
	\item if t = m: return true
	\item else: return false
	\end{enumerate}

\section{main files and command line inputs}\label{ss:main}
Key Generator
\begin{enumerate}
\item -b = minimum bits for mod(n)
\item -i = number of iterations to test prime numbers. Default 50
\item -n pbfile = specifies file that has public key. Default rsa.pub
\item -d pvfile = specifies file that has private key. Default rsa.priv
\item -s = specifies random seed for random state. Default time(NULL)
\item -v  = verbose output
\item -h = synopsis and usage
\item set file permissions to 0600 using fchmod and fileno
\item fileno returns int, so store it in a variable and use it for fchmod
\item fchmod(fileno integer, 0600 permissions)
\item randstate init(s)
\item make public and private keys
\item getenv(USER) to get username as string
\item convert username using mpz set str() with a base of 62
\item write public key to pbfile and private key to pvfile
\item if verbose output:
	\begin{enumerate}
	\item sizeinbase(mpz, base2) to get bit numbers
	\item print username with newline
	\item print signature s with newline
	\item print prime p with newline
	\item print prime q with newline
	\item print mod(n) with newline
	\item print exponent e with newline
	\item print private key d with newline
	\item each of these lines should have number of bits for each and the decimal value that correspons to them
\item randstate clear() and close/clear all files and variables
\end{enumerate}
Encrypt
\begin{enumerate}
\item -i = input file to encrypt. Default = stdin
\item -o = output file to encrypt to. Default = stdout
\item -n = file containing public key. Default = rsa.pub
\item -v = verbose output
\item -h = synopsis and usage
\item open file and exit program if there is a problem opening the file
\item read public key from pbfile
\item if verbose:
	\begin{enumerate}
	\item print username with newline
	\item print signature s with newline
	\item print mod(n) with a newline
	\item print exponent e with a newline
	\item print each with their number of bits and their values as a decimal
	\end{enumerate}
\item convert username to mpzt using set str() and verify it using rsa verify() 
\item encrypt file using rsa encrypt file()
\item close pbfile and clear mpz variables
\end{enumerate}
Decrypt
\begin{enumerate}
\item -i = input file to decrypt. Default = stdin
\item -o = output file to decrypt to. Default = stdout
\item -n = specifies file containing private key. Default = rsa.priv
\item -v = verbose output
\item -h = synopsis  and usage
\item open private key file. Print error if failed
\item read private key from pvfile
\item if verbose:
	\begin{enumerate}
	\item print mod(n) with newline
	\item print private key e with newline
	\item both should print number of bits and their values in decimal
	\end{enumerate}
\item decrypt file using rsa decrypt file()
\item close pvfile and clear mpz variables
\end{enumerate}

\end{document}
