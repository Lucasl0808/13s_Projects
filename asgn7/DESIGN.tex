\documentclass[11pt]{article}
\title{cse13s asgn7 DESIGN.pdf}
\author{Lucas Lee; CruzID: luclee}
\date{3/3/2022}
\begin{document}\maketitle
\section{Program Details}\label{ss:details}
We will be using the k-nearest neighbors algorithm to identify the most likely author of an anonymous text.
\section{Manhattan Distance}\label{ss:manhattan}
The Manhattan Distance is one of the 3 distance metrics we will use. The main concept for this is that you get the sum of the frequency/total of each word in one vector to those of another.
\begin{enumerate}
\item Comparing 2 vectors, get the fractional component of each, and get the absolute value of that number
\item Repeat this for each vector component and get the fractional sum of the values.
\item The value of the summation of these numbers is the Manhattan Distance
\end{enumerate}

\section{Euclidean Distance}\label{ss:euclidean}
The Euclidean Distance is the default of the 3 metrics we will use. This calculation subtracts the vector components and squares them before getting the total summation, then takes the square root of taht summation
\begin{enumerate}
\item Get the difference of two vector components
\item Square the difference from the step above
\item Repeat this for all vector components
\item Sum up all values from the calculation and take the square root of the number to get the Euclidean Distance
\end{enumerate}

\section{Cosine Distance}\label{ss:Cosine}
The Cosine Distance is the last of the 3 distance metrics. This calculation multiplies the vector components and subtracts the sum of these numbers from 1 to get the Cosine Distance.
\begin{enumerate}
\item Get the product of two vector components
\item Store the sum of each of the products
\item Repeat for each of the vector components
\item Subtract the counting sum of the products from 1 to get Cosine Distance.
\end{enumerate}

\section{Hash Table}\label{ss:Hashtable}
We will need to make a Hash Table struct in order to store unique words in a text with the amount to times it is used. Hash tables allow for us to look for these words at a potentially O(1) time complexity.
\begin{enumerate}
\item HashTable htcreate(uint32t size)
	\begin{enumerate}
	\item constructs a hash table.
	\item Set size equal to parameter size
	\item Allocate memory for array of Nodes with capacity of parameter size
	\item salts.h has the data for salt in the HashTable parameter
	\end{enumerate}
\item void htdelete(HashTable ht)
	\begin{enumerate}
	\item destroys the hash table.
	\item Free all nodes in the hash table
	\item After freeing the nodes, set the pointer to ht to NULL
	\end{enumerate}
\item uint32t htsize(Hashtable ht)
	\begin{enumerate}
	\item return the total number of slots in the hash table
	\item this is given by the size used when creating the hashtable
	\end{enumerate}
\item Node htlookup(HashTable ht, char word)
	\begin{enumerate}
	\item Searches given hash table for a node that contains the word given as a parameter
	\item Return the pointer to the node if found, othewise return a NULL pointer
	\end{enumerate}
\item Node htinsert(HashTable ht, char word)
	\begin{enumerate}
	\item Insert word into given hash table.
	\item If the word is already in the table, increment its count by 1.
	\item Use Linear Probing to traverse the hash table and find a location for the word.
	\item If Linear Probing fails because hash table is full, Return a NULL pointer.
	\item If the Node is successfully inserted, return the pointer to that Node
	\end{enumerate}
\item void htprint(HashTable ht)
	\begin{enumerate}
	\item debug function used to print the contents of the hash table.
	\item write this immediately after htcreate() to help debug each function.
	\end{enumerate}
\end{enumerate}

\section{Hash Table Iterator}\label{ss:iterator}
Create a struct to assist in iterating over the hash table. This should be created in the same file as the constructor and the other hash table functions.
\begin{enumerate}
\item HashTableIterator hticreate(HashTable ht)
	\begin{enumerate}
	\item Constructor for the hash table iterator.
	\item Slot parameter will be initialized to 0.
	\item HashTable parameter will be set to the given ht parameter
	\end{enumerate}
\item void htidelete(HashTableIterator hti)
	\begin{enumerate}
	\item Delete the current iterator.
	\item Do not delete the hash table that corresponds with this iterator.
	\end{enumerate}
\item Node htiter(HashTableIterator hti)
	\begin{enumerate}
	\item Return the pointer to the node at the next entry in the hash table.
	\item Incrementing the slot parameter of the iterator may assist in this.
	\item If the next entry is not valid(there is nothing there) increment the slot parameter again.
	\item Return NULL if the iterator slot is at the max hash table size (iterated over entire table).
	\end{enumerate}
\end{enumerate}

\section{SPECK Cipher}\label{ss:speck}
Using a given SPECK Cipher, we are given a hash function that takes a salt(2 uint64ts) and a key to hash. The hash() function returns a uint32t which is the index where the key is mapped.

\section{Nodes}\label{ss:nodes}
We will need to create nodes that contain a word and its frequency(count).
\begin{enumerate}
\item Node nodecreate(char word)
	\begin{enumerate}
	\item Create a copy of the word that is passed into this function as a parameter using strdup()
	\item Allocate memory for the size of the word and copy the word into char parameter
	\end{enumerate}
\item void nodedelete(Node n)
	\begin{enumerate}
	\item Destroy a node by freeing the memory for word, and then free the node.
	\item After memory is freed, set the node pointer to NULL
	\end{enumerate}
\item void nodeprint(Node n)
	\begin{enumerate}
	\item Debug function that prints out the contents of the node
	\item Do this function quickly to test if node is functioning properly
	\end{enumerate}
\end{enumerate}

\section{Bloom Filters}\label{ss:bloom}
We will be using Bloom Filters to determine if a word is most likely in the hashtable. The struct for bloom filter will be using 3 different salts to give an estimate of the word.
\begin{enumerate}
\item BloomFilter bfcreate(uint32t size)
	\begin{enumerate}
	\item Set each of the salts in the struct to the given corresponding salts in salts.h.
	\item Allocate memory for the bit vector filter that should have the necessary amount of bits for the Bloom filter.
	\end{enumerate}
\item void bfdelete(BloomFilter bf)
	\begin{enumerate}
	\item Destructor for the Bloom filter, freeing memory allocated for the bitvector and any other memory allocated in the constructor.
	\item Set pointer to NULL after freeing all necessary memory.
	\end{enumerate}
\item uint32t bfsize(BloomFilter bf)
	\begin{enumerate}
	\item Return the number of bits that BloomFilter bf can access.
	\item This is also the length of bit vector
	\end{enumerate}
\item void bfinsert(BloomFilter bf, char word)
	\begin{enumerate}
	\item Inserts word into the bloom filter.
	\item Uses hash() on the word for each salt in the bloom filter. 
	\item Sets the bit at the indices of the bit vector
	\end{enumerate}
\item bool bfprobe(BloomFilter bf, char word)
	\begin{enumerate}
	\item Run hash() on the word for each salt in the bloom filter similar to bfinsert().
	\item Check if the bits at the indices are set, and return true if they are all set.
	\item If not all the bits at those indices are set, return false.
	\end{enumerate}
\item void bfprint(BloomFilter bf)
	\begin{enumerate}
	\item Debug function that prints out the bits of Bloom filter.
	\item This may use the print function for debugging in bit vector code.
	\end{enumerate}
\end{enumerate}

\section{Bit Vectors}\label{ss:bitvector}
We need to make a Bit Vector ADT that represents a 1D array of bits. This will be used to indicate true and false through checking if a given bit within a byte is a 1 or 0. Bit shifting logic from code.c in assignment 6 can applicable here.
\begin{enumerate}
\item BitVector brcreate(uint32t length)
	\begin{enumerate}
	\item Constructor for the bit vector.
	\item Return NULL in the event that sufficient memory cannot be allocated.
	\item Otherwise, return a BitVector $\star$ or a pointer to a BitVector that has allocated space.
	\item Initialize each bit of the vector parameter to 0.
	\end{enumerate}
\item void bvdelete(BitVector bv)
	\begin{enumerate}
	\item Destructor for the bit vector.
	\item Free memory allocated with the BitVector bv passed in as a parameter.
	\item After freeing the memory, set the pointer of bv to NULL.
	\end{enumerate}
\item uint32t bvlength(BitVector bv)
	\begin{enumerate}
	\item Return the length of the bit vector.
	\item This is given by the value in the struct length
	\end{enumerate}
\item bool bvSetBit(BitVector bv, uint32t i)
	\begin{enumerate}
	\item Sets the bit at index i to 1.
	\item Return false if I is out of range and true if the bit is successfully set.
	\end{enumerate}
\item bool bvClrBit(BitVector bv, uint32t i)
	\begin{enumerate}
	\item Clear bit at index i by setting it to 0.
	\item If I is out of range, return false.
	\item Return true to indicate successful clearing of a bit.
	\end{enumerate}
\item bool bvGetBit(BitVector bv, uint32t i)
	\begin{enumerate}
	\item Gets the bit at index i.
	\item Return false if i is out of range.
	\item Return true if the value at index i is 1.
	\item Return false if the value at index i is 0.
	\end{enumerate}
\item void bvprint(BitVector bv)
	\begin{enumerate}
	\item Debug function that prints out the bits in the bit vector.
	\item Iterates over each byte in the bitvector and individually prints each bit.
	\end{enumerate}
\end{enumerate}

\section{Regular Expressions, Parser.c, regex}\label{ss:regex}
We will have files that allow us to use Regular Expressions to find words, contractions and hyphenations.

\begin{enumerate}
\item The given parser.h and parser.c have functions to look for words using nextword().
\item This function takes in a file to read words, and a regular expression.
\item Use regcomp() to compile the regular expression to pass it into the funciton.
\end{enumerate}

\section{Text}\label{ss:text}
We will use a Text struct ADT to keep track of how many words are in each text.
\begin{enumerate}
\item Text textcreate(FILE infile, Text noise)
	\begin{enumerate}
	\item Constructor for text.
	\item Get each word using functions in parser.c and regex. Convert all words to lowercase.
	\item Each lowercased word that is not in the noise parameter is added to the created Text.
	\item If the Text noise is NULL, then the Text that is created in this function will be the noise.
	\item If you cannot allocate sufficient memory then the function returns NULL.
	\item Otherwise, Return a Text pointer or pointer to an allocated Text.
	\item Hash table should be created with a size of \(2^{19}\), and the Bloom Filter should be created with a size of \(2^{21}\).
	\end{enumerate}
\item void textdelete(Text text)
	\begin{enumerate}
	\item Delete a text by freeing hash table, bloom filter, then the text itself.
	\item After frees, set pointer to NULL.
	\end{enumerate}
\item double textdist(Text text1, Text text2, Metric metric)
	\begin{enumerate}
	\item Returns the distance between text1 and text2 using one of the 3 metrics specified at the beginning of the document.
	\item Metric.h provides the metric parameter.
	\item NOTE: nodes contain counts for their words, and need to normalize it with the total word count in the text
	\end{enumerate}
\item double textfrequency(Text text, char word)
	\begin{enumerate}
	\item Return frequency of parameter word in parameter text. 
	\item If word is not in the text, return 0.
	\item Otherwise return the normalized frequency of word.
	\end{enumerate}
\item bool textcontains(Text text, char word)
	\begin{enumerate}
	\item Return true or false based on if the word is inside the given text or not.
	\item True if word is in text, false if not.
	\end{enumerate}
\item void textprint(Text text)
	\begin{enumerate}
	\item Debug function that prints the contents of parameter text.
	\item Possible to call other debug functions of the other components of text to debug here.
	\end{enumerate}
\end{enumerate}

\section{Priority Queue}\label{ss:pq}
We will need to implement a priority queue similar to assignment 6's priority queue struct.

\begin{enumerate}
\item PriorityQueue pqcreate(uint32t capacity)
	\begin{enumerate}
	\item Constructor for priority queue.
	\item Priority queue should have an author and the corresponding distance associated with that author as struct parameters.
	\item Parameter capacity sets the total capacity of the priority queue.
	\item If memory cannot be allocated return NULL from this function.
	\item Otherwise, return PriorityQueue pointer or pointer to allocated PriorityQueue.
	\item Priority queue should be initialized to have 0 elements inside the queue.
	\end{enumerate}
\item void pqdelete(PriorityQueue q)
	\begin{enumerate}
	\item Destructor for PriorityQueue q.
	\item Free all memory associated with PriorityQueue q.
	\item All remaining objects in the queue that are leftover must be removed/freed.
	\item Set the q pointer to NULL after freeing all memory.
	\end{enumerate}
\item bool pqempty(PriorityQueue q)
	\begin{enumerate}
	\item Return true if PriorityQueue q is empty, and false if there is more than 1 element in the queue.
	\item It may help and make it easy to do this if a size parameter is added to the struct.
	\end{enumerate}
\item bool pqfull(PriorityQueue q)
	\begin{enumerate}
	\item Return true if PriorityQueue q is full, and false if there are 0 elements in the queue.
	\item If a size parameter is added, this will be if the size is equal to the capacity.
	\end{enumerate}
\item uint32t pqsize(PriorityQueue q)
	\begin{enumerate}
	\item Return the number of elements in PriorityQueue q.
	\item If a size parameter is added, this will just be the value of the size parameter.
	\end{enumerate}
\item bool enqueue(PriorityQueue q, char author, double dist)
	\begin{enumerate}
	\item Return false if the PriorityQueue q is full.
	\item Enqueue the pair of an author and the distance dist corresponding to the author to the queue.
	\item Return true to indicate the successful enqueue of the author, distance pair.
	\item Queue should be sorted with highest priority going to the author with the lowest value for dist.
	\end{enumerate}
\item bool dequeue(PriorityQueue q, char author, double dist)
	\begin{enumerate}
	\item Return false if the PriorityQueue q is empty.
	\item Dequeue an author/dist pair from the queue, shifting every element in the queue over by 1 to accomodate.
	\item Pass the char author back into the double pointer for author, and pass the distance back with the pointer to dist.
	\item Return true to indicate the successful dequeue of a pair.
	\end{enumerate}
\item pqprint(PriorityQueue q)
	\begin{enumerate}
	\item Debug function for checking if the Priority queue is implemented correctly.
	\end{enumerate}
\end{enumerate}

\section{Identify(Main function)}\label{ss:main}
We will make an identify.c function that allows us to find the most likely author of a given text passed into stdin.

\begin{enumerate}
\item Handle and account for command line options:
	\begin{enumerate}
	\item -d = Specify path to database of authors and texts(default = lib.db).
	\item -n = Specify path to file of noise words (default = noise.txt).
	\item -k = Specify number of authors to match to the given text (default = 5).
	\item -l = Specify number of noise words (default = 100).
	\item -e = Set metric to use as Euclidean distance (default metric).
	\item -m = Set metric to use as Manhattan distance.
	\item -c = Set metric to use as Cosine distance.
	\item -h = Displays help and usage.
	\item NOTE: for setting the metrics, it should disable the other metrics if selected.
	\end{enumerate}
\item Specifics for main function:
	\begin{enumerate}
	\item Create a new Text from stdin. This will be the text we will try to identify the author for.
	\item Create Text for the noise file. This file is specified in the command line inputs. Make all words in noise lowercased.
	\item Open database for authors and texts specified in command line -d. The first line in the file specifies the number of author/text pairs, we will call this value n.
	\item Create a PQ that holds n elements. This makes it so that we can look through all the authors in the database.
	\item Read in the rest of the file to get the authors and their texts.
		\begin{enumerate}
		\item Use fgets() once to get the author, and use it again to get the path to the text for that author.
		\item The call to fgets() keeps the newline characters, so we will need to remove it.
		\item Open the author's text and create a new Text with the file.
		\item If the file fails to open, then skip over it and continue reading the database.
		\item If the file opens, make sure that the parsed words are lowercased before adding the word to Text
		\item Use the author's text and the anonymous text from stdin to compute a distance after filtering noise words.
		\item Enqueue that author's name and the corresponding distance to the PQ.
		\end{enumerate}
	\item After reading through the entire database, we should have a PQ of the most likely authors for the text.
	\item Dequeue from PQ k number of times to get the most likely authors.
	\end{enumerate}
\end{enumerate}

\end{document}

